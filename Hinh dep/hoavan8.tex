%\documentclass[magick={density=600,outfile=\jobname.png}]{standalone}
\documentclass[tikz,border=2mm]{standalone}
\usetikzlibrary{calc}
\begin{document}
	
	% màu hoa hồng Bungary
	\definecolor{bulgarianrose}{rgb}{0.28, 0.02, 0.03}
	
	\begin{tikzpicture}
		\def\a{2} % nửa cạnh hình vuông
		\pgfmathsetmacro{\r}{sqrt(2)*\a} % bán kính cung tròn
		\pgfmathsetmacro{\d}{(2*sqrt(2)-2)*\a} % độ dài chân miền tô màu
		
		% các tọa độ và nhãn ban đầu
		\coordinate (O) at (0,0);
		\coordinate [label=above left:{\textbf A}] (A) at (-\a,\a);
		\coordinate [label=above right:{\textbf B}] (B) at (\a,\a);
		\coordinate [label=below right:{\textbf C}] (C) at (\a,-\a);
		\coordinate [label=below left:{\textbf D}] (D) at (-\a,-\a);
		
		% tô màu 4 miền con
		\foreach \i/\mau in {0/lime,90/red!80,180/brown!80,270/cyan!80}
		\filldraw[draw=bulgarianrose, fill=\mau, rotate=\i]
		(O) arc(-45:0:\r) -- ++(-\d,0) arc(-180:-135:\r)--cycle;
		
		% nối các nét vẽ
		\draw[color=bulgarianrose, line width=1pt, line join=round, line cap=round] (A)--(B)--(C)--(D)--cycle;
		\draw[bulgarianrose, opacity=.5] (A)--(C) (B)--(D);
		
		\draw (O) node[below=.1cm] {\textbf O};
	\end{tikzpicture}
\end{document}