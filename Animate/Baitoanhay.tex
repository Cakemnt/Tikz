\documentclass[tikz,border=2mm]{standalone}
\usepackage{tikz}
\usetikzlibrary{calc,intersections,angles,quotes}
\usepackage{animate}
\begin{document}
%==================
\def\sfa{100}
\begin{animateinline}[controls,autoplay,palindrome,loop]{8}
\multiframe{\sfa}{i=1+1}{
%\foreach \i in {0,...,\sfa}{
\begin{tikzpicture}
\pgfmathsetmacro\tl{\i/\sfa}
\path
(0,0) coordinate (x)
--++(10,0) coordinate (x')
(0,-3) coordinate (z)
--++(10,0) coordinate (z')
($(x)!\tl!(z)$) coordinate (y)
($(x')!\tl!(z')$) coordinate (y')
($(x)!.4!(x')$) coordinate (A)
($(y)!(A)!(y')$) coordinate (I)
($(A)!1!60:(I)$) coordinate (K)
($(K)!.1!90:(A)$) coordinate (Kt)
(intersection of K--Kt and z--z') coordinate (C)
($(A)!1!-60:(C)$) coordinate (B)
($(z)!(A)!(z')$) coordinate (H)
($(C)!2!(H)$) coordinate (C_1)
($(A)!1!60:(C_1)$) coordinate (B_1)
;
\foreach \i in {x,y,z}\draw (\i) node[left]{$\i$}--(\i') node[right]{$\i'$};
\draw[dashed] 
(A)--(I)--(K) --cycle (Kt)--(C) (I)--(H)
(A) let\p1=($(A)-(C)$) in circle ({veclen(\x1,\y1)})
;
\draw 
(A)--(B)--(C)--cycle
(A)--(B_1)--(C_1)--cycle
pic[draw,angle radius=2mm]{right angle =A--I--y}
pic[draw,angle radius=2mm]{right angle =A--K--C}
;
\foreach \x/\g in {A/90,B/135,C/-90,I/-45,K/-45,H/-90,B_1/45,C_1/-45} \draw[fill=white] (\x) circle (.03) + (\g:.3) node{$\x$};
\path
(current bounding box.south west) node[below right,align=left]{
{\bfseries Dựng tam giác đều có ba đỉnh $A$, $B$, $C$ lần lượt trên ba đường thẳng song song $xx'$, $yy'$, $zz'$} \\
B1: Lấy điểm $A$ bất kỳ trên $xx'$. Lấy $I$ trên $yy'$ sao cho $AI\perp yy'$. \\
B2: Quay $I$ quanh $A$ một góc $60^{\circ}$ được $K$. \\
B3: Qua $K$ vẽ đường vuông góc với $AK$ cắt $zz'$ tại $C$. \\
B4: Vẽ đường tròn tâm $A$ bán kính $AC$ cắt $yy'$ tại $B$. \\
Bài toán có hai nghiệm hình đối với mỗi điểm $A$.
};
\end{tikzpicture}}
\end{animateinline}
\end{document}