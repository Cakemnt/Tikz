\documentclass[11pt,border=2mm]{standalone}
\usepackage[utf8]{vietnam}
\renewcommand{\familydefault}{\sfdefault}
\usepackage{amsmath,amscd,commath,mathrsfs}
\usepackage{graphicx,tikz,tkz-tab}
\usetikzlibrary{arrows,patterns,calc,intersections,angles,quotes}
\usetikzlibrary{decorations.markings,backgrounds}
\usepackage{tkz-euclide,anyfontsize}
\usetkzobj{all}
\begin{document}
	\begin{tikzpicture}[font=\fontsize{7}{1}\selectfont ,scale=3]
		%\draw[gray,thin,opacity=.5] (-2,-2) grid (2,2);
		%====== trái đất và hai cung trong
		\definecolor{col2}{rgb}{0,1,0.6}
		\def\a{0.18}
		\def\b{0.1}
		\fill[name path =c2,fill=col2,opacity=.5] (0,0) circle(\a cm);
		\draw[rotate=-23.5, red] (\a,0) arc(0:-180:\a cm and \b cm);
		\draw[rotate=-23.5, red,dashed] (\a,0) arc(0:180:\a cm and \b cm);
		%========= ellipse ngang
		\def\aa{1}
		\def\bb{0.3}
		\draw[dashed,blue,thick,name path =ar1] (\aa,0) arc(0:180:\aa cm and \bb cm);
		\draw[blue,thick,name path =ar2] (\aa,0) arc(0:-180:\aa cm and \bb cm);
		%=========ellipse chéo
		\draw[dashed,red,thick,rotate=-23.5,name path =ar3] (\aa,0) arc(0:180:\aa cm and \bb cm);
		\draw[red,thick,rotate=-23.5,name path=ar4] (\aa,0) arc(0:-180:\aa cm and \bb cm);
		%===========vòng tròn ngoài
		\definecolor{col1}{rgb}{0,0.5,1}
		\shade[name path = c1,ball color=col1,opacity=.6] (0,0) circle(1cm);
		%=============== các điểm vàng
		\fill[yellow](\aa,0) circle(0.8pt);
		\fill[yellow](-\aa,0) circle(0.8pt);
		\path[name intersections={of =ar1 and ar3,by=X}]; % bắc
		\fill[yellow](X) circle(.8pt);
		\path[name intersections={of =ar2 and ar4,by=Y}]; % nam
		\fill[yellow](Y) circle(.8pt);
		%======= Các cung 23.5
		\coordinate (A) at ($(0,0)+(-60:\aa cm and \bb cm)$);
		\pgfmathsetmacro{\q}{sqrt(3)/2}
		\draw[<->] (A) arc(0:-14:\q) node[right,pos=.5,align=center]{$23.5^\circ$} ;
		\coordinate (B) at ($(0,0)+(120:\aa cm and \bb cm)$);
		\draw[<->] (B) arc(180:166:\q) node[left,pos=.5,align=center]{$23.5^\circ$} ;
		\draw[<->] (1.04,0) arc(0:-47:1.04 cm);
		\node[right] at (1.04,0) {$(+)$};
		\node[right] at (-47:1.04) {$(-)$};
		%========== Các điểm thiên cực
		\coordinate[label=above right: Thiên cực bắc] (C) at (68:\aa cm);
		\coordinate[label=below left: Thiên cực nam] (D) at (248:\aa cm);
		\draw [dashed] (D)--(C);
		\draw (C)--($(D)!1.1!(C)$) (D) -- ($(C)!1.1!(D)$);
		\draw[<-] (X) -- (-0.1,1.1) node[above]{Thu phân} ;
		\draw[<-] (X) -- (-0.8,0.9) node[above,align=center]{Mặt trời \\ trong năm};
		\draw[<-] (-1,0) --(-1.2,0.3) node[above,align=center]{Đông chí};
		\draw[<-] (Y) --(0.2,-1.1) node[below,align=center]{Xuân phân};
		\draw[<-] (1,0) --(1.2,0.3) node[above,align=center]{Hạ chí};
		\node at (1.1,-0.5) {Xích vĩ};
		\draw[<-] ($(-160:\aa cm and \bb cm)$)--(-1.15,-0.4) node[below]{Hoàng đạo};
		\draw[<-] (-0.8,0.48) --(-1.2,0.6)node[above]{Quỹ đạo trời};
		\draw[<-] ($(D)!0.8!(C)$) --(0.9,0.62)node[right]{Địa trục};
		\fill(-23.5:1.04) circle(0.4pt) node[right]{$0^\circ$};
	\end{tikzpicture}
\end{document}