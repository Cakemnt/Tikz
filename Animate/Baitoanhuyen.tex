\documentclass{standalone}
\usepackage[utf8]{vietnam}
\usepackage{tikz}
\usetikzlibrary{calc,intersections}
\usepackage{animate}
\begin{document}
%==================
\begin{animateinline}[controls,autoplay,palindrome,loop]{8}
\multiframe{40}{i=1+1}{
%\foreach \k i {1,...,40,41,40,...,1}{
\begin{tikzpicture}
\def\r{1}%Bán kính đường tròn (O)
\path
(0:0) coordinate (A) %Điểm A cố định
(\i:0.5) coordinate (d)% Điểm d để xác định đường thẳng d
(0,4) coordinate (O)%Tâm đường tròn O
($(O)+(\i+90:\r)$) coordinate (M') %Điểm M'
($(O)+(\i-90:\r)$) coordinate (N')%Điểm N'
(\i+90:3) coordinate (u)
;
\path[name path =AM] (A)--($(A)!1.5!(M')$);
\path[name path=AN] (A)--($(A)!2!(N')$);
\path[name path=tron] (O) circle (\r);
\path[name intersections={of =AM and tron}] (intersection-2) coordinate(M);
\path[name intersections={of =AN and tron}] (intersection-1) coordinate (N);
\path
(intersection of O--M and A--u) coordinate (O1)
(intersection of O--N and A--u) coordinate (O2)
;
\clip (-6,-2) rectangle (6,6);
\draw[dashed] ($(u)!2!(A)$)--($(A)!2!(u)$) (O)--(O1) (N)--(O2);
\draw 
(O) circle (\r)
(A)--(M')
(A)--(N)
;
\draw($(A)-(\i:3)$)--($(d)+(\i:3)$);
\draw (O1) let \p1=($(M)-(O1)$), in circle ({veclen(\x1,\y1)});
\draw (O2) let \p1=($(N)-(O2)$), in circle ({veclen(\x1,\y1)});
\foreach \x in {A,O,M',N',M,N,O1,O2}{\draw[fill=white] (\x) circle (1pt);}
\path 
(A) node[below left] {A}
(O) node[below left] {O}
(M') node[above]{M'}
(N')node[below]{N'}
(M)node[above left]{M}
(N)node[below right]{N}
(O1)node[left]{O$_{1}$}
(O2)node[left]{O$_{2}$}
;
\end{tikzpicture}
}
\end{animateinline}
\end{document}