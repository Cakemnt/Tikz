%\documentclass[tikz,border=5mm,convert={outfile=\jobname.png}]{standalone}
\documentclass{standalone}
\usepackage[utf8]{vietnam}
\usepackage{tikz}
\usetikzlibrary{calc}
\usepackage{animate}
\usepackage{ifthen}
\begin{document}
\def\hinhvuong(#1,#2){
\draw (#1,#2) rectangle (#1+1,#2+1);
\draw[fill=red] (#1,#2+0.75) rectangle (#1+0.25,#2+1);
\draw[fill=blue] (#1,#2) rectangle (#1+0.25,#2+0.25);
\draw[fill=orange] (#1+0.75,#2) rectangle (#1+1,#2+0.25);
\draw[fill=yellow] (#1+0.75,#2+0.75) rectangle (#1+1,#2+1);
}
\def\n{30}
\def\m{7}
\pgfmathsetmacro{\mm}{\m-1}
\pgfmathsetmacro{\cao}{div(\n,\m)+1.5}
%==================
\begin{animateinline}[controls,autoplay,loop]{4}
\multiframe{\n}{i=0+1}{
%\foreach \i in {1,...,200}{
\pgfmathsetmacro{\k}{div(\i,\m)}
\pgfmathsetmacro{\h}{mod(\i,\m)}
\begin{tikzpicture}
\clip (0,-0.5) rectangle (\m+0.5,\cao);
\ifthenelse{\k<1}{\foreach \x in {0,...,\h}{\hinhvuong(\x,0)}}{}
\ifthenelse{\k >0}{\pgfmathsetmacro{\km}{\k-1}
\foreach \y in {0,...,\km}{\foreach \x in {0,...,\mm}{\hinhvuong(\x,\y)}}
\foreach \x in {0,...,\h}{\hinhvuong(\x,\k)}
}{}
\end{tikzpicture}
}
\end{animateinline}
\end{document}