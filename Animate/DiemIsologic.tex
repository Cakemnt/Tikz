\documentclass{standalone}
\usepackage{tikz}
\usetikzlibrary{calc,intersections}
\usepackage{animate}
\begin{document}
%==================
\def\tamngoaitiep(#1,#2,#3)(#4){
\path
($(#1)!.5!(#2)$) coordinate (#1#2)
($(#1)!.5!(#3)$) coordinate (#1#3)
($(#1#2)!1mm!90:(#1)$) coordinate (#1#2t)
($(#1#3)!1mm!90:(#1)$) coordinate (#1#3t)
(intersection of #1#2--#1#2t and #1#3--#1#3t) coordinate (#4)
;
}
\def\chanphangiac(#1,#2,#3)(#4){
\path
($(#2)!1mm!(#1)$) coordinate (#2#1)
($(#2)!1mm!(#3)$) coordinate (#2#3)
($(#2#1)!.5!(#2#3)$) coordinate (#2tpg)
(intersection of #2--#2tpg and #1--#3) coordinate (#4)
;
}
\def\tructam(#1,#2,#3)(#4){
\path
($(#1)!(#2)!(#3)$) coordinate (#1#2#3)
($(#2)!(#3)!(#1)$) coordinate (#2#3#1)
(intersection of #2--#1#2#3 and #3--#2#3#1) coordinate (#4)
;
}

%\pagecolor{green}
\def\r{3}
\begin{animateinline}[controls,autoplay,palindrome,loop]{8}
\multiframe{50}{i=0+1}{
%\foreach \i in {0,...,49,50,49,...,0}{
\begin{tikzpicture}[join=round,cap=round,transform shape]
\path
(0:0) coordinate (O)
(95+\i:\r) coordinate (A)
(210:\r) coordinate (B)
(-30:\r) coordinate (C)
($(B)!.5!(C)$) coordinate (L)
($(C)!.5!(A)$) coordinate (M)
($(A)!.5!(B)$) coordinate (N)
(L) coordinate (X)
(M) coordinate (Y)
(N) coordinate (Z)
($(Y)!.5!(Z)$) coordinate (Txy)
($(X)!.5!(Z)$) coordinate (Txz)
;
%%%%%%%%%%%%
\tamngoaitiep (X,Y,Z)(Oxyz)
\chanphangiac(Z,X,Y)(X_1)
\chanphangiac(X,Y,Z)(Y_1)
\path
($(X)!1mm!90:(X_1)$) coordinate (Xt)
($(Y)!1mm!90:(Y_1)$) coordinate (Yt)
(intersection of X--Xt and Y--Z) coordinate (X_2)
(intersection of Y--Yt and X--Z) coordinate (Y_2)
($(X_1)!2!(Txy)$) coordinate (X_1')
($(X_2)!2!(Txy)$) coordinate (X_2')
($(Y_1)!2!(Txz)$) coordinate (Y_1')
($(Y_2)!2!(Txz)$) coordinate (Y_2')
($(X_1')!.5!(X_2')$) coordinate (tamx)
($(Y_1')!.5!(Y_2')$) coordinate (tamy)
;
\path[name path=cirax] (tamx) let\p1=($(tamx)-(X_1')$) in circle ({veclen(\x1,\y1)});
\path[name path=ciray] (tamy) let\p1=($(tamy)-(Y_1')$) in circle ({veclen(\x1,\y1)});
\path[name intersections={of=cirax and ciray}] (intersection-2) coordinate (P);
%%%%%%%%%%%%
\path
($(A)!(O)!(P)$) coordinate (Oa)
($(B)!(O)!(P)$) coordinate (Ob)
($(C)!(O)!(P)$) coordinate (Oc)
($(A)!2!(Oa)$) coordinate (D)
($(B)!2!(Ob)$) coordinate (E)
($(C)!2!(Oc)$) coordinate (F)
;
\tamngoaitiep(A,P,E)(O_1)
\tamngoaitiep(A,P,F)(O_2)
\tamngoaitiep(B,P,F)(O_3)
\tamngoaitiep(B,P,D)(O_4)
\tamngoaitiep(C,P,D)(O_5)
\tamngoaitiep(C,P,E)(O_6)
\tamngoaitiep(O_1,O_2,O_3)(tamo)
%%%%%%%%%%%%
\pgfresetboundingbox
\clip (O) circle (1.5*\r);
\draw[red] (O_1) let\p1=($(O_1)-(P)$) in circle ({veclen(\x1,\y1)});
\draw[red] (O_2) let\p1=($(O_2)-(P)$) in circle ({veclen(\x1,\y1)});
\draw[red] (O_3) let\p1=($(O_3)-(P)$) in circle ({veclen(\x1,\y1)});
\draw[red] (O_4) let\p1=($(O_4)-(P)$) in circle ({veclen(\x1,\y1)});
\draw[red] (O_5) let\p1=($(O_5)-(P)$) in circle ({veclen(\x1,\y1)});
\draw[red] (O_6) let\p1=($(O_6)-(P)$) in circle ({veclen(\x1,\y1)});
\draw[blue] (tamo) let\p1=($(tamo)-(O_1)$) in circle ({veclen(\x1,\y1)});
\draw[red]
(O) circle (\r)
;
\draw 
(A)--(B)--(C)--cycle 
(A)--(D) (B)--(E) (C)--(F)
;
\foreach \x/\g in {A/90,B/180,C/0,O/90,P/70}\draw[blue,fill=white] (\x) circle (.05) +(\g:.3) node{$\x$};
\foreach \x/\g in {D/-90,E/0,F/180,O_1/90,O_2/120,O_3/180,O_4/240,O_5/-90,O_6/0}\draw[fill=white] (\x) circle (.05) +(\g:.3) node{$\x$};
\end{tikzpicture}
}
\end{animateinline}
\end{document}
