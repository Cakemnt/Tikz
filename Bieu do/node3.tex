\documentclass[tikz, border=5pt]{standalone}
\usetikzlibrary{shadows}
\usepackage[utf8]{vietnam}
\usepackage{amsmath}
\tikzstyle{mynodeone}=[
align=center,
anchor=center,
rounded corners=4pt,
drop shadow=gray]
\tikzstyle{mynodetwo}=[%shape=cloud,
below,
minimum height=1cm,
minimum width=4cm,
align=center,
anchor=center,
rounded corners=4pt,
drop shadow=gray]
\tikzstyle{mynodethree}=[below,
rounded corners=4pt,
drop shadow=gray]
\begin{document}
	\mathversion{bold}
	\begin{tikzpicture}[ultra thick, cap=round,join=round]
		\path(0,0)node[mynodeone,fill=orange]
		(a){$AX=B$, $B\neq \theta$};
		\path(a.south)--+(3,-2)
		node[mynodetwo,
		outer color=red!80!white,
		inner color=red!20!white](a1)
		{Hệ vô nghiệm\\
			$\mathrm{rank}(A)\neq\mathrm{rank}(A|B)$
		};
		\path(a.south)--+(-3,-2)
		node[mynodetwo,
		outer color=cyan!80!white,
		inner color=cyan!20!white](a2)
		{Hệ có nghiệm\\
			$\mathrm{rank}(A)=\mathrm{rank}(A|B)$
		};
		\path(a2.south)--+(3,-1)
		node[mynodethree,
		outer color=green!80!white,
		inner color=green!20!white](a21)
		{\begin{minipage}
				{4 cm}
				Hệ có vô số nghiệm\\
				$\underbrace{\mathrm{rank}(A)=\mathrm{rank}(A|B)}_{=r< n \text{ (số ẩn)} }$\\
				Nghiệm của hệ phụ thuộc $n-r$ tham số.
			\end{minipage}
		};
		\path(a2.south)--+(-3,-1)
		node[mynodethree,
		outer color=yellow!80!white,
		inner color=yellow!20!white](a22)
		{\begin{minipage}
				{4cm}
				Hệ có nghiệm duy nhất\\
				$\underbrace{\mathrm{rank}(A)=\mathrm{rank}(A|B)}_{= n \text{ (số ẩn)} }$
			\end{minipage}
		};
		%vẽ các đường mũi tên
		\draw[->](a)--++(-90:1)-|(a1.north);
		\draw[->](a)--++(-90:1)-|(a2.north);
		%
		\draw[->](a2)--++(-90:1)-|(a21.north);
		\draw[->](a2)--++(-90:1)-|(a22.north);
	\end{tikzpicture}
\end{document}