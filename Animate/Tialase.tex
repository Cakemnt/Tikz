\documentclass{standalone}
\usepackage{tikz}
\usetikzlibrary{calc,intersections}
\usepackage{animate}
\begin{document}
%==================
\def\m{3}%Rộng
\def\n{3}%Cao
\def\k{4}%Dài
\def\sfa{100}
\begin{animateinline}[controls,autoplay,palindrome,loop]{16}
\multiframe{\sfa}{i=0+1}{
%\foreach \i in {0,1,...,99,100,99,...,0}{
\begin{tikzpicture}[join=round,cap=round]
%\def\glp{30}
\pgfmathsetmacro\tl{\i/\sfa}
\pgfmathsetmacro\glp{20+\tl *50}
\path
(0:1) 
--++(180+\glp:.5*\m) coordinate (B)
--++(0:\k)--++(\glp:.5*\m)--++(90:\n) coordinate (A)
;
\clip (-\m,-\m) rectangle (\k+2,\n+\m);
\draw[ultra thick, red] (A)--($(A)!\tl!(B)$);
\foreach \x in {1,...,\k}{
\foreach \y in {1,...,\n}{
\begin{scope}[shift={(\x,\y)}]
\draw 
(0:0)--(180+\glp:.5*\m)
(0:0)--++(-90:1)--++(180+\glp:.5*\m)
;
\foreach \l in {0,...,\m}\draw (180+\glp:.5*\l)--++(-90:1)--++(0:1);
\fill[blue!50,opacity=.5] (0:0)--(180+\glp:.5*\m)--++(0:1)--++(\glp:.5*\m)--cycle;
\fill[purple!50,opacity=.5] (180+\glp:.5*\m)--++(0:1)--++(-90:1)--++(180:1)--cycle;
\fill[brown!50,opacity=.5] (0:1)--++(180+\glp:.5*\m)--++(-90:1)--++(\glp:.5*\m)--cycle;
\foreach \l in {0,...,\m}\draw (180+\glp:.5*\l)--++(0:1)--++(-90:1);
\draw (0:1)--++(180+\glp:.5*\m)--++(-90:1)--++(\glp:.5*\m);
\end{scope}
}
}
\fill[ball color=blue] (B)node[below left]{$B$} circle (.025);
\fill[ball color=blue] (A) node[above right]{$A$} circle (.025);
\end{tikzpicture}
}
\end{animateinline}
\end{document}
