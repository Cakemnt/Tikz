\documentclass{standalone}
\usepackage{tikz}
\usetikzlibrary{calc}
\usepackage{animate}
\begin{document}
	%==================
	\def\n{180}
	\begin{animateinline}[controls,autoplay,palindrome,loop]{8}
		\multiframe{\n}{i=0+2}{
			%\foreach \i in {0,2,...,178,180,178,...,0}{
				\begin{tikzpicture}
					\pgfmathsetmacro\xrad{\i *pi/180}
					\clip (0,-2) rectangle (4*pi,2);
					\draw[dashed](0,0)--(4*pi,0);
					\draw[thick,red,smooth,samples=100] plot[domain=0:4*pi](\x,{-1.5 *sin((\x-\xrad) r) }) ;
					\foreach \j in {1,2,...,32}{
						\pgfmathsetmacro\x{\j *pi/8}
						\pgfmathsetmacro\y{-1.5*sin((\x-\xrad) r)}
						\draw[red,-latex](\x,0)--(\x,\y);
					}
					\foreach \k in {0,...,3}{
						\pgfmathsetmacro\xm{\k *pi+pi/3}
						\pgfmathsetmacro\xh{\xm+pi/3}
						\pgfmathsetmacro\ym{(-1.5*sin((\xm -\xrad) r))/3}
						\pgfmathsetmacro{\yh}{-1.5*sin((\xh -\xrad) r)/3}
						\fill[ball color=violet!70!white] (\xm,\ym)node{$+$ } circle (pi/12);
						\fill[ball color=violet!70!white] (\xh,\ym)node{$+$ } circle (pi/12);
						\fill[ball color=red!50!white] (\k *pi +pi/2,0)node{$=$ } circle (pi/6);
					}
				\end{tikzpicture}
			}
		\end{animateinline}
	\end{document}