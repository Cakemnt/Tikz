\documentclass{standalone}
\usepackage{tikz}
\usepackage[utf8]{vietnam}
\usepackage{animate}
\usetikzlibrary{calc,intersections}
\begin{document}
%==================
\def\n{100}
\def\hsf(#1){(#1)^3+3*(#1)^2-2}
\begin{animateinline}[controls,autoplay,palindrome,loop]{16}
\multiframe{\n}{i=0+1}{
%\foreach \i in {0,...,99,100,99,...,0}{
\begin{tikzpicture}[join=round,cap=round]
\pgfmathsetmacro\m{2.5-\i/20}
\clip (-4,-3.5) rectangle (3.5,3); 
\draw[-stealth] (-3.5,0)--(3,0) node[below left]{$x$ };
\draw[-stealth] (0,-3)--(0,3) node[below left]{$y$ };
\draw[dashed] (-2,0)|-(0,2);
\draw[smooth,name path=f] plot[domain=-3.1:1.075] (\x,{\hsf(\x)}) (1,{\hsf(1)})node[above right]{$y=f(x)$ };
\draw[thick,blue,name path=m] (-3.5,\m)--(3,\m) node[above left]{$y=m$ } (0,\m) node[above left]{$m$ };
\fill[ball color=green] (0,\m) circle (.03);
\path 
(0,0) node[above right]{$O$ }
(0,2) node[above right]{$2$ }
(0,-2) node[below left]{$-2$ }
(-2,0) node[below]{$-2$ }
;
\ifnum \i<10
\path[name intersections = {of= m and f,by={A}}];
\fill[ball color=orange] (A) circle (.05);
\path (0,-3) node[below]{$m>2$: $1$ giao điểm $\Rightarrow 1$ nghiệm};
\else
\ifnum \i=10
\path[name intersections = {of= m and f,by={A}}];
\fill[ball color=orange] (A) circle (.05);
\fill[ball color=orange] (-2,2) circle (.05);
\path (0,-3) node[below]{$m=2$: $2$ giao điểm $\Rightarrow 2$ nghiệm};
\else
\ifnum \i<90
\path[name intersections = {of= m and f,by={A,B,C}}];
\fill[ball color=orange] (A) circle (.05);
\fill[ball color=orange] (B) circle (.05);
\fill[ball color=orange] (C) circle (.05);
\path (0,-3) node[below]{$-2<m<2$: $3$  giao điểm $\Rightarrow 3$ nghiệm};
\else
\ifnum \i=90
\path[name intersections = {of= m and f,by={A}}];
\fill[ball color=orange] (A) circle (.05);
\fill[ball color=orange] (0,-2) circle (.05);
\path (0,-3) node[below]{$m=-2$: $2$  giao điểm $\Rightarrow 2$ nghiệm};
\else
\path[name intersections = {of= m and f,by={A}}];
\fill[ball color=orange] (A) circle (.05);
\path (0,-3) node[below]{$m<-2$: $1$  giao điểm $\Rightarrow 1$ nghiệm};
\fi
\fi
\fi
\fi
\end{tikzpicture}
}
\end{animateinline}
\end{document}
